\documentstyle{article}

\title{Implementation of Scheme}

\author{Jason Coughlin}
\date{3 May 1990}

\begin{document}

\maketitle

\section{Introduction}

	Scheme is a properly tail-recursive, statically-scoped dialect of
LISP.  Scheme's data-types include symbols, numbers, characters, strings,
lists, vectors, functions, and continuations.  Its syntax and evaluation
rules are simple and consistent, allowing Scheme programmers to easily mix
many programming paradigms.

\section{Function naming}

	The implementation described in this paper was done in the C
programming language.  Care was taken to keep the implementation
modularized.  Currently the modules are:

	\begin{enumerate}
	\item scanner ({\tt scanner.c}) -- the lexical scanner
	\item microcode ({\tt micro.c}) -- low-level operations that
directly manipulate representations
	\item the symbol table ({\tt symstr.c}) -- a closed hash table of
the symbols
	\item operations ({\tt ops.c}), predicates ({\tt preds.c}), forms
({\tt forms.c}) -- the Scheme primitive functions, predicates, and
special forms
	\item the evaluator ({\tt eval.c}) -- the interpreter itself
	\item the top-level ({\tt scheme.c})
	\end{enumerate}

	A distinction must be made between microcode and operations.
Microcode ({\tt micro.c}) contain low-level operations required to
implement the language and data structures.  The microcode directly
manipulates structures, creating lists, symbols, and strings.  It also
performs searching, arithmetic, and input/output.  By calling microcode,
the operations ({\tt ops.c}) implement Scheme functions; there is one
operation for each Scheme function supported by the implementation.

	The names of microcode routines sometimes overlap the names of
operations.  A consistent and easy to remember naming scheme is to prefix
each routine with the module to which it belongs.  For example, the
microcode {\tt assoc} is named {\tt mcAssoc} while the operation {\tt
assoc} is named {\tt opAssoc}.

\section{Data Representation}

	In most imperative languages like Pascal and C, indentifiers are
{\bf statically typed}.  That is, the programmer explicitly declares the type of
each variable to be an integer, character, or string.  In contrast, Scheme
is {\bf dynamically typed}, meaning the type of a variable dynamically changes
at run-time.

	In this implementation, all of Scheme's data-types (lists,
numbers, characters, strings, vectors) are stored in one higher-level
representation called {\bf cons} nodes.  A cons node may contain an
integer, or a float, or a character, etc.  Obviously, the actual
representations of these types are different, so a cons node is defined as
one structure with a field for each of these data-types.  As a cons node
can only be one type at any one time, it is actually the union of these
types for space efficiency.

\begin{verbatim}
/* the cons node datatype */
struct Pair {
        struct C *car;
        struct C *cdr;
} ;
struct C {
   int cell_type;               /* PAIR, INT, FLOAT, SYMBOL, CHAR, etc. */
   BOOL mmark;

   union {
        struct Pair cons_node;
        int int_data;
        REAL_NUM float_data;
        char *symbol;           /* also points to the string */
        char character;
   } data;
} ;
typedef struct C CONSNODE;
typedef struct C *CONS;
\end{verbatim}

	Each cons node represents one symbol, number, character, or
string.  The {\tt cell\_type} field marks the type of a cons node.  In the
implementation, the macro {\tt mcKind} is used to reference the {\tt
cell\_type} field of a given cons node.  For example, {\tt mcKind(c) =
PAIR} sets the {\tt cell\_type} field of cons node {\tt c} to {\tt PAIR}.

\section{Memory Management}

	The Scheme programmer is free of the burden of explicitly
deallocating memory for cons nodes.  This implementation of Scheme
allocates a certain amount of memory for cons nodes and uses this memory
until it is gone.  When it needs a new cons node, Scheme {\bf
garbage-collects}, gathering all cons nodes that aren't currently being
used.

	This implementation of Scheme uses the Mark-Sweep garbage
collection method.  A {\bf segment} with a fixed amount of cons nodes is
allocated, and the nodes are placed on a free-list.  Cons nodes are then
taken off the free-list on demand.  When the free-list is empty, the
garbage-collector marks all nodes being used (by marking the stacks and
the current environment) then sweeps through the segments gathering all
nodes that weren't marked, placing them on the free-list.  If all of the
nodes are currently being used, then the free-list is still empty after a
garbage-collection.  More cons nodes are needed so a new segment is
allocated and placed on the free-list.

	The marking algorithm passes through the current Scheme
environment, marking all Scheme variables and their values ({\bf NOTE:}
The Scheme environment is discussed in the next section).  This protects a
Scheme programmer's variables from being garbage-collected.  However, a
mechanism is needed to protect the implementor's local variables as well.
This implementation uses a {\bf register stack}.  The register stack is an
array of pointers to cons nodes, and it is manipulated using these macros:

\begin{verbatim}
/* MM macros */
#define ENTER		CONS *Old_Top = Top_RegS
#define REG(x)		CONS *x = (mcRegPush( NIL ), Top_RegS)
#define R(x)		(*(x))
#define LEAVE		Top_RegS = Old_Top; return
\end{verbatim}

	On entry, a C function with local Scheme variables uses the {\tt
ENTER} macro to mark the current placement in the register stack.  The
programmer then declares the Scheme variables using the {\tt REG} macro;
these variables are called {\bf registers} (although they shouldn't be
confused with C's concept of a register variable).  Since the registers
are actually pointers to cons nodes, the programmer manipulates register
variables using the {\tt R} macro.  Lastly, the programmer uses the {\tt
LEAVE} macro to actually return from this C procedure.  {\tt LEAVE}
restores the register stack to its state before {\tt ENTER} was called.
Here is a short example:

\begin{verbatim}
/* mcCons(head, tail) - Returns a cons node whose car = head and cdr = tail. */
CONS mcCons(head, tail)
CONS head, tail;
{
   ENTER;               /* mark current placement of register stack */
   REG(h);              /* declare h, t, and res as cons variables */
   REG(t);
   REG(res);

   R(h) = head;         /* h = head of pair */
   R(t) = tail;         /* t = tail of pair */

   /* need a cons node */
   R(res) = new_cons();
   mcKind( R(res) ) = PAIR;     /* result is a cons node of type PAIR */

   /* build the cons node */
   mcGet_Car( R(res) ) = R(h);
   mcGet_Cdr( R(res) ) = R(t);

   LEAVE R(res);      /* restore the register stack and return res */
}
\end{verbatim}

	Now, to protect local C variables, the programmer declares them
as register variables, and the marking algorithm is modified so it will
mark the register stack.  After the C procedure calls {\tt LEAVE}, the
local variables are popped off the register stack and will be
garbage-collected.

\section{Static Scope}

	All Scheme bindings of identifiers to values are stored in the
{\bf environment}.

	Scheme Versions 1.0 and 1.1 used association lists to represent
environments.  An {\bf association list} (sometimes abbreviated as an {\bf
a-list}) is a list of pairs.  The a-list is of the form {\tt ( (symbol1 .
value1) ... )}.  When evaluating a symbol, the interpreter used the Scheme
operation {\tt assoc} to find the pair that corresponds to the symbol and
returned its {\tt cdr}.  {\tt assoc} examines the a-list from left to
right, returning the first pair with a matching symbol.  Therefore, the
worst case time of finding the binding of an identifier was O(N) where N
is the number of bindings (the length of the a-list).

	Scheme 1.2 uses a different representation.  In Scheme 1.2, an
environment is composed of two pieces: a vector and an a-list.  The vector
is used to store global bindings (created with the {\tt define} special
form) while the a-list (called the nested environment) is used to store
bindings of parameters and nested scopes (created with the {\tt lambda}
special form).  To lookup a symbol in the vector is constant time and to
lookup a symbol in the a-list is still O(N).  However, since the a-list
only contains parameters and nested scopes, N will be quite small.

	Note that since Scheme is statically scoped, the compiler could
be modified to generate code to lookup parameters in constant time.
However, the compiler in version 1.2 does not use this optimization.

	The Scheme special-form {\tt define} is used to define global
variables; the symbols are said to be bound at the top-level.  To bind a
variable to a value, the value is placed in the appropriate vector
element.  In this implementation, it is an error to {\tt define} a symbol
that is already bound.

	To change the value of a variable, the special form {\tt set!} is
called with the variable name and the new value.  {\tt set!} simply
changes the appropriate vector element.

	In Scheme, functions are {\bf first-class objects} which means
that functions can be stored in variables, passed as arguments to other
functions, and returned as values.  Functions are defined with the {\tt
lamdba} special form.  The syntax of {\tt lambda} is:

\begin{center}
	{\tt (lambda (parm1 ...) exp1 exp2 ...)}
\end{center}

	Note that unlike imperative languages which bind all functions to
a name, this is an ``anonymous'' function; it has no name associated with
it.  To bind a function to a name, the {\tt define} and {\tt lambda}
special forms are combined.

\begin{center}
	{\tt (define (name parm1 ...) exp1 exp2 ...)}
\end{center}

	Actually, this is ``syntactic sugar'' as a Scheme interpreter
typically converts this expression into this form:

\begin{center}
	{\tt (define name (lambda (parm1 ...) exp1 exp2 ...))}
\end{center}

	Evaluation of the {\tt lambda} special form returns a {\bf
closure}.  Since functions are first-class objects, a closure is a
data-type and can be bound to variables as described above.  To achieve
static scope, the closure contains the nested portion of the environment
in effect when the function was defined.  The parameters and body of the
function are also stored in the closure which is represented with an
extension to the cons node:

\begin{verbatim}
/* defn of user-defined closures */
struct Closure {
   struct C *env;          /* env when closure was defined */
   struct C *parms;        /* parameters to closure */
   struct C *body;         /* body of procedure */
} ;
struct C {
   int cell_type;               /* PAIR, INT, FLOAT, SYMBOL, CHAR, etc. */
   BOOL mmark;

   union {
        struct Pair cons_node;
        struct Closure close;
        int int_data;
        REAL_NUM float_data;
        char *symbol;           /* also points to the string */
        char character;
   } data;
} ;
typedef struct C CONSNODE;
typedef struct C *CONS;
\end{verbatim}

	When a closure is applied to its arguments, the parameters to the
closure are bound to their arguments over the nested environment {\em in the
closure}.  This achieves {\bf static scope} as the environment in the
closure is the environment that was in effect when the closure was
defined.  Bindings are ``consed'' onto the front of this environment.
Note that global variables might exist with the same name as a parameter,
so the lookup routine first checks the nested portion of the environment
then checks the vector portion.

\section{Scheme's Evaluation Rules}

	A {\bf pair} is a cons node that points to two other cons nodes.
An atom is any data-type that is not a pair.  Symbols, characters,
numbers, strings, and closures are examples of atoms.  In Scheme, the
empty list {\tt ()} is also an atomic data-type; it is not considered to
be a pair so the empty list is also an atom.

	An expression is either an atom or a pair of expressions.

	The interpreter sits in a {\bf read-evaluate-print} loop,
abbreviated as the {\bf REP} loop.  Scheme reads in an expression,
evaluates it, and prints the value.

   \subsection{Evaluating atoms}

	Characters, numbers, and strings evaluate to themselves.

\begin{verbatim}
   ]=> #\a
   #\a

   ]=> 12.5
   12.5

   ]=> "this is a TEST"
   "this is a TEST"
\end{verbatim}

	Symbols evaluate to the value to which they are currently bound.
It is an error to evaluate an unbound symbol.  Assume {\tt a} is bound to
5, {\tt b} is bound to the string ``this is a test'', and {\tt c} is an
unbound variable.

\begin{verbatim}
   ]=> a
   5

   ]=> b
   "this is a test"

   ]=> c
   ERROR: Unbound symbol c.  Returning to top-level.
\end{verbatim}

	Here is a function which will evaluate an atom:

\begin{verbatim}

/* evEvalAtom( atom, env ) - Returns the evaluation of the atom in the
	current evironment.
*/
static CONS evEvalAtom( atom )
CONS atom;
{
   CONS binding;

   /* (eval #T) => #T && (eval #F) => #F && (eval number) => number */
   if ( atom == T  || atom == F || mcNumber(atom) )
	return atom;

   /* (eval string) => string && (eval char) => char */
   if ( mcString(atom) || mcChar(atom) || mcVector(atom) )
	return atom;

   if ( !mcSymbol(atom) )
	RT_LERROR("EVAL: Can't evaluate non-symbol: ", atom);

   /* (eval symbol) => symbol's-binding */
   binding = evAccNested( atom, glo_env );
   if ( !mcNull(binding) ) {
	/* symbol's-binding = cdr(binding) [remember they're dps!] */
	binding = mcCdr( binding );
	return binding;
   }
   else if ( (binding = evAccGlobal( atom, glo_env )) != NULL ) {
	return binding;
   }

   RT_LERROR("EVAL: Undefined symbol ", atom);
}
\end{verbatim}

   \subsection{Evaluating a list of expressions}

	Scheme uses prefix notation when evaluating a list of
expressions; the first element of the list is considered to be the
function or special form to be applied and the rest of the elements are
the function's or form's arguments.

	First, Scheme evaluates the first element of the list to
determine whether it is a special form or a function.  If it is a special
form, the evaluator immediately invokes the special form with its
arguments unevaluated.  If the first element is a function, the evaluator
recursively evaluates each of the function's arguments before applying
the function to its arguments.

	The evaluator is actually broken into two pieces, an evaluator
({\tt eval}) and a function applicator ({\tt apply}).  The function
applicator takes a {\em closure} and its arguments and actually invokes
the function on its arguments.  The evaluator is then defined in terms of
the applicator.  {\tt eval} evaluates the expressions of a list to obtain
the closure and its evaluated arguments.  It then invokes the applicator
with the closure and its arguments.  Since the applicator receives its
arguments already evaluated, special forms must be handled by the
evaluator itself.

\section{First stage: Recursive Model}

	Since Scheme's evaluation rule is recursively defined, it makes
sense that it can be implemented using recursion.  In the recursion model,
{\tt eval} will be handed an expression to evaluate and the environment in
which to evaluate it.  The expression can be either an atom or a list.  If
the expression is an atom, {\tt eval} calls {\tt mcEvalAtom} with the atom
to be evaluated and the environment passed to eval.  If the expression is
a list, {\tt eval} evaluates the first element of the list to determine
whether or not it is evaluating a special form or a function.

	If {\tt eval} is evaluating a special form, it invokes the special
form with its unevaluated arguments and the current environment.  The
environment is necessary as some forms like {\tt define} and {\tt set!}
evaluate some of their arguments.  In this model, these forms simply call
{\tt mcEval} with the argument needing evaluation and the environment
which was passed to them.  Notice that since special forms evaluate their
arguments in the {\em current} environment, they have {\bf dynamic scope}.

	However, if {\tt eval} is evaluating a function, it calls itself
on each of the function's arguments, collecting their values into a list.
To invoke the function on its arguments, it simply calls {\tt apply} with
the closure, the evaluated argument list, and the current environment.

\begin{verbatim}
CONS evEval(expr, env)
CONS expr, env;
{
   if ( mcAtom(expr) )
        /* symbols, numbers, characters, strings, and '() */
        return evEvalAtom( expr, env );
   else {
        ENTER;
        REG(f);
        REG(e);
        REG(args);
        REG(val);

        /* evaluating a list of expressions -- evaluate first element */
        R(f) = evEval( mcCar(expr), env );
        expr = mcCdr(expr);

        /* is it a special form? */
        if ( mcForm( R(f) ) )
                /* invoke form on it's arguments */
                return evEvalForm( R(f), expr, env );

        /* evaluate the arguments
         * reverse the arguments
         * (arg1 arg2 ...) becomes (... arg2 arg1)
         * then args are evaluated and consed together back into the right
         * order
         */
        R(args) = NIL; R(e) = mcRev(expr);
        while ( !mcNull( R(e) ) ) {
                /* evaluate arg and tack onto front of arg list */
                R(args) = mcCons( mcEval( mcCar( R(e) ), env ), R(args) );
                R(e) = mcCdr( R(e) );
        }

        /* apply the function to its args */
        R(val) = evApply( R(f), R(args) );
        MCLEAVE R(val);
   }
}
\end{verbatim}

	Now the only missing piece is {\tt apply}, the function
applicator.  {\tt apply} takes a closure, its arguments, and the current
environment.  It returns the value of ``applying'' the closure to its
arguments in the current environment.  The first step to applying a
procedure is to extend the environment in the closure by binding the
parameters of the procedure to its arguments (Refer to {\bf Static
Scope}).  Then the body of the procedure is evaluated with this extended
environment, and the value of the last expression is returned.  When {\tt
apply} is done, the environment in effect {\em before} the application
must be restored so that the parameter bindings disappear.

\begin{verbatim}
/* evApply(func, args) - Apply the function to its arguments in the
        function's environment.  func is a closure and args is the list of
        evaluated arguments.
*/
CONS evApply(func, args)
CONS func, args;
{
   ENTER;
   REG(body);           /* the body of the procedure */
   REG(env);            /* the environment to evaluate the body in */
   REG(value);          /* value of last evaluation */

   /* make sure we're applying a procedure */
   if ( !mcClosure(func) )
        RT_LERROR("APPLY: Attempt to apply non-procedure: ", func);

   /* (1) Extend environment; Bind parameters to args.
    *     bindings are consed onto the front of the environment
    *     in the closure to achieve static scope
    */
   R(env) = evBindArgs( mcGet_Parms(func), args, mcGet_Env(func) );

   /* (2) Evaluate the body in this closure */
   R(body) = mcGet_Body(func);
   R(value) = NIL;

   while ( !mcNull( R(body) ) ) {
        /* evaluate the first expression in the extended environment */
        R(value) = evEval( mcCar( R(body) ), R(env) );

        /* evaluate the rest of the expressions */
        R(body) = mcCdr( R(body) );
   }

   /* return the value -- recursion will automatically restore the previous
    * environment for us.  the extended environment will be garbage-
    * collected next garbage-collection
    */
   MCLEAVE R(value);
}
\end{verbatim}

	Notice that {\tt eval} and {\tt apply} are mutually recursive.  It
is the implementation language's (C's) recursion that will
``automatically'' restore the previous environments as we step back
through the recursion to the top-level.

	To conclude, this model of the interpreter is fairly easy to
understand and implement.  It is a good model to aquaint the implementor
with the semantics of evaluation.  However, it is not possible to
implement Scheme's more powerful constructions with this model in C.

\section{Second stage: The Stack Model}

	The interpreter must constantly keep track of the current state of
evaluation.  A state of evaluation is both the expression to evaluate {\em
and} where to pass the value of the evaluation.  Passing the value of an
evaluation is completing a computation and is called the {\bf
continuation}.  Essentially, a continuation is how to {\em continue} a
computation.  Scheme allows the Scheme programmer to ``capture'' any
evaluation's continuation and to ``continue'' the computation later.

	In the recursive-descent model, the implementation language's
recursion is implicitly handling the state of the interpreter for the
programmer.  This might be fine for a simple dialect of LISP, since the
programmer has such limited control over any state.  However, it is not a
suitable model for Scheme.  Continuations grant the Scheme programmer
explicit control of evaluation.  Clearly, the implementor needs explicit
control over the state of the interpreter.  The programmer gains this
control by using a stack-based model.

	Any state of evaluation has two parts, the expression to evaluate,
and what to do with the value.  Two stacks are used to represent any
particular state, an expression stack and a value stack.  The expression
stack, denoted e-stack, will contain the expressions to evaluate while the
value stack, denoted v-stack, will hold the values of the evaluations.
Normally, the evaluator will pop an expression off the e-stack, evaluate
it, and push the value on the v-stack.  Evaluation continues until the
e-stack is empty.

	This model is completely iterative and needs to be iterative so
the implementator has complete control over evaluation.  It is easy to
evaluate atoms.  However, we learned in the last section that lists are
evaluated by recursively evaluating their sub-expressions.

	To recursively evaluate the sub-expressions of a list, {\tt eval}
pops the list off the e-stack and pushes each element of the list back on
the e-stack.  On the next iteration, {\tt eval} will start popping and
evaluating the sub-expressions.  The only problem is recognizing that all
of a function's arguments have been evaluated and that it is time to
apply the function.  The solution is to push markers on both stacks
before the sub-expressions are pushed.  When a marker is popped off the
e-stack, the function and its arguments are on the v-stack.  They are
popped, the arguments are consed into a list, and {\tt apply} is called.

	The sub-expressions need to be pushed onto the e-stack so that the
first sub-expression is evaluated first to determine whether it is a
special form or a function.

   \subsection{Special Forms}

	Most special forms evaluate some of their arguments.  In the
recursive model, a special form just called {\tt mcEval} with the
expression to be evaluated.  Under the stack model, the programmer can not
use recursion; the expression must be pushed on the e-stack and the
special form must return to {\tt mcEval} which will evaluate the
expression and leave the result on the v-stack.  The problem is that the
special form needs to terminate in order to evaluate an argument, but the
form can't terminate until it has completed its work.  A continuation is
needed to ``continue'' the special form {\em after} its argument has been
evaluated.  This will be discussed later, but the cost of an actual
continuation is too high to just ``resume'' a special form.  Instead,
this implementation uses the concept of a {\bf resume}.

	The special form is broken into two procedures.  One procedure
will perform any work that can be done before an argument needs to be
evaluated.  It will then package the work left to do into a resume.  A
resume is much simpler than a continuation.  It is a list of the form:
{\tt (*RESUME* special-form anything-needed-by-the-special-form)} that is
pushed on the e-stack underneath the argument to be evaluated.  This
procedure then terminates.  {\tt mcEval} pops the argument, evaluates it,
pushes the value on the v-stack, and pops the resume.  Since the resume
contains the special form, {\tt mcEval} is able to call the second
procedure of the special form.  The second procedure simply pops the value
of the argument off the v-stack and completes the work.

   \subsection{Restoring environments}

	The recursive model passed a ``current environment'' in which to
evaluate expressions.  Since the environments were ``local'', the
recursion automatically restored environments.  For the iterative stack
model, it is more convienent for the ``current environment'' to be the
global variable {\tt glo\_env}.  Restoring the previous environment is now
the programmer's responsibility, and it is handled by saving {\tt glo\_env}
before extending the environment in the closure to be evaluated.

	Remember that the e-stack contains
``expressions-to-be-evaluated.''  Therefore, {\tt apply} can push {\tt
glo_env} on the e-stack followed by a special marker called {\tt RESTORE}.
Then to evaluate the closure, {\tt glo\_env} is set to the extension of
environment in the closure  The body of the closure is pushed on the
e-stack, and {\tt apply} simply returns to {\tt eval}.  {\tt eval} pops
the body of the closure off the e-stack and evaluates it, leaving the
value on the v-stack.  Finally, the previous environment is restored when
{\tt eval} pops {\tt RESTORE}.

   \subsection{Implementation}

\begin{verbatim}
/* evEval() - Evaluate the expression on TOP of the expression stack in the
        given environment.  Returns the evaluation on the value stack.
        Expressions are evaluated in the current environment which is
        glo_env (a global variable).
*/
void evEval()
{
   ENTER;

   REG(exp);            /* expression to evaluate */
   REG(func);           /* function to invoke */
   REG(eargs);          /* evaluated args */
   REG(tmp);            /* temporary register */

   R(eargs) = NIL;      /* no args evaluated yet */

   /* Keep evaluating until there are no more expressions to evaluate */
   while ( mcHaveExprs() ) {

        /* evaluate the expression on top of the expression stack */
        R(exp) = mcPopExpr();

        if ( R(exp) == CALL ) {

                /* apply the function on the value stack */

                /* gather the arguments and function off of the
                 * value stack.
                 */
                R(eargs) = evGatherArgs();
                R(func) = mcCar( R(eargs) );
                R(eargs) = mcCdr( R(eargs) );

                /* apply func to it's arguments */
                evApply( R(func), R(eargs) );
        }
        else if ( R(exp) == RESTORE ) {
                /* restore previous environment */
                glo_env = mcPopExpr();
        }
        else if ( mcResume( R(exp) ) ) {
                /* resume a func or form */
                evInvokeRes( R(exp) );
        }
        else if ( mcAtom( R(exp) ) ) {

                /* evaluate an atom */
                R(exp) = evEvalAtom( R(exp) );

                /* invoke a special form before it's args are
                 * evaluated.
                 */
                if ( mcForm( R(exp) ) )
                        evInvokeForm( R(exp) );
                else mcPushVal( R(exp) );
        }
        else {
                /* evaluate list of expressions */

                /* mark place on stacks */
                mcPushVal( MARK );
                mcPushExpr( CALL );

                /* put exps on stack, so stack-top == 1st exp
                 * mcRev() is destructive so need to keep a pointer
                 * so we can restore the exp later
                 */
                R(exp) = mcRev( R(exp) );
                R(tmp) = R(exp);

                /* NOTE: DO-WHILE assumes (CAR NIL) => NIL */
                do {
                        mcPushExpr( mcCar( R(exp) ) );
                        R(exp) = mcCdr( R(exp) );
                } while ( R(exp) != NIL );

                /* restore exp to original state */
                R(tmp) = mcRev( R(tmp) );
        }
   }
   MCLEAVE;
}

/* evApply(func, args) - Apply the function to its arguments in the
        function's environment.  func is a closure and args is the list of
        evaluated arguments.
*/
CONS evApply(func, args)
CONS func, args;
{
   /* make sure we're applying a procedure */
   if ( !mcClosure(func) )
        RT_LERROR("APPLY: Attempt to apply non-procedure: ", func);

   /* (1) Save the current environment */
   mcPushExpr( glo_env );
   mcPushExpr( RESTORE );

   /* (2) Extend environment; Bind parameters to args.
    *     bindings are consed onto the front of the environment
    *     in the closure to achieve static scope
    */
   glo_env = evBindArgs( mcGet_Parms(func), args, mcGet_Env(func) );

   /* (3) Evaluate the body in this closure */
   mcPushExpr( mcGet_Body(func) );
}
\end{verbatim}

   \subsection{Tail Recursion}

	A {\bf tail-call} occurs when a procedure directly calls another
procedure as its value.  A procedure is {\bf tail-recursive} when it
recursively tail-calls itself.

\begin{verbatim}
   (define (write-out n)
     (if (zero? n)
         (write n)
         (write-out (- n 1))
     )
   )
\end{verbatim}

	The procedure {\tt write-out} is tail-recursive as its value is
the value returned by directly calling itself with {\tt n-1}.  Calling
{\tt (write-out 100)} in most dialects of LISP would push 100 {\tt
RESTORE}s and 100 previous environments to backtrack through before the
value of {\tt (write 0)} would be returned.  However, Scheme is properly
tail-recursive, which means that Scheme treats tail-recursive calls as
simple branches.  A Scheme implementation will only push one environment
and one {\tt RESTORE}.

	Implementation of tail-call handling is straight-forward in the
stack-model.  A tail-call involves the direct application of another
procedure as the value of the current procedure.  To invoke the tail-call,
{\tt apply} will push a {\tt RESTORE}.  But since the tail-call is the
value of the current procedure, there are no more expressions to the body
of the current procedure on the e-stack.  This means that the {\tt
RESTORE} that {\tt apply} pushed before invoking the current procedure is
now on the top of the stack.  To eliminate tail-recursion, {\tt apply}
simply checks to make sure that a {\tt RESTORE} is not on the top of the
e-stack.

	A tail-recursive procedure will cause only one environment to be
saved on the e-stack, the environment before the {\em first} invokation of
the tail-recursive procedure.  As a result, the stack will not grow very
large during evaluation, and the tail-recursive procedure will immediately
return to its caller without backtracking through the recursive calls.
The net effect is that tail-recursive procedures become simple loops.

   \subsection{Continuations}

	Recall that any state of the interpreter involves what to evaluate
and what to do with the value.  ``What to do with the value'' is called
the {\bf continuation} of an evaluation.  The Scheme programmer can
capture the continuation of an evaluation and ``continue'' with a
different value with the {\tt call-with-current-continuation}, abbreviated
as {\tt call/cc}, procedure.

\begin{center}
{\tt (call/cc procedure)}
\end{center}

	{\tt call/cc} obtains {\em its} continuation (where the call to
{\tt call/cc} is supposed to pass its value) and passes this continuation
to {\tt procedure}.  {\tt procedure} must be a function of one argument.
Since the continuation represents ``what to do with the value'', the
continuation itself is a function of one argument, the value to continue
with after the {\tt call/cc} call.  Here are some examples taken from
\underline{The Scheme Programming Language} by R. Kent Dybvig.

\begin{verbatim}
   ]=> (call/cc procedure?)

   #T
\end{verbatim}

	This example illustrates the point that a continuation acts like a
function of 1 argument.  {\tt procedure?} is a function of 1 argument that
returns {\tt #T} if it's argument is a function, {\tt #F} otherwise.  {\tt
call/cc} captures it's continuation and binds it to {\tt procedure?}'s
parameter.  {\tt procedure?} returns that the continuation is a function.

\begin{verbatim}
   ]=> (call/cc
           (lambda (return-value)
               (* 5 4)))
   20
\end{verbatim}

	In this example, the {\tt call/cc}'s continuation is bound to
{\tt return-value}.  However, the lambda function never uses {\tt
return-value} so the continuation is never invoked.  The value of {\tt
call/cc} in this case is the value of the function: 5 * 4.

\begin{verbatim}
   ]=> (call/cc
           (lambda (return-value)
               (* 5 (return-value 4))))
   4
\end{verbatim}

	Again, {\tt call/cc}'s continuation is bound to {\tt
return-value}.  However, in this example, {\tt return-value} is {\em
invoked} with the value 4.  Therefore, the {\tt call/cc}'s evaluation
continues with the value 4.  The part of the expression ``(* 5'' is never
even evaluated.

\begin{verbatim}
   ]=> (+ 2
            (call/cc
                (lambda (return-value)
                   (* 5 (return-value 4)))))
   6
\end{verbatim}

	This example is just like the previous example accept the value
of the continuation (4) is added to 2.

	The state of the interpreter is represented with the current
e-stack, v-stack, and environment.  Therefore, to capture the current
state, a continuation needs to capture the current e-stack, v-stack, and
environment.  Here is the structure representing a continuation:

\begin{verbatim}
/* defn of a continuation */
struct Continuation {
   struct C *env;       /* env in effect at capture */
   struct C *vals;      /* value stack at capture */
   struct C *exps;      /* expression stack at capture */
} ;
\end{verbatim}

	When the interpreter evaluates {\em the continuation}, it pops the
continuation's argument off the v-stack, restores e-stack, v-stack, and
environment from the continuation, and pushes the value back on the
v-stack.  This effectively continues evaluation with the specified value
from the point when the continuation was captured.

	Remember that certain special forms needed to evaluate some of
their arguments.  Since the interpreter is iterative, they needed to push
their argument, return to {\tt eval}, then ``continue'' their own work.
Special forms could be implemented with continuations.  However, capturing
and restoring the e-stack and v-stacks is an expensive operation,
especially when the interpreter has a lot to evaluate.  Using ``resumes''
accomplishes the same task without the overhead associated with
continuations.

\end{document}
